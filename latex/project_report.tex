\documentclass[conference]{IEEEtran}
\usepackage{enumitem}
\usepackage[document]{ragged2e}
\usepackage{blindtext}
\usepackage{graphicx}
\usepackage{float}
\usepackage{amsmath}
%\usepackage{natbib}
\usepackage{cite}
\usepackage{lipsum}

\graphicspath{ {./}{./images/} }

\title{Joint Radar Communication System using OFDM}

\author{

\IEEEauthorblockN{Owen Sowatzke}
\IEEEauthorblockA{\textit{Electrical Engineering Department} \\
\textit{University of Arizona}\\
Tucson, USA \\
osowatzke@arizona.edu}

\and
\IEEEauthorblockN{Iman Miraki}
\IEEEauthorblockA{\textit{Electrical Engineering Department} \\
\textit{University of Arizona}\\
Tucson, USA \\
imanmiraki@arizona.edu}}

\begin{document}


	\raggedbottom
	\maketitle
	
\begin {abstract}


This paper presents the design and evaluation of a Joint Radar-Communication (JRC) system utilizing Orthogonal Frequency Division Multiplexing (OFDM). The proposed system combines radar and communication functionalities to enhance spectrum efficiency, reduce hardware costs, and improve resource sharing. A zero-forcing technique is employed to generate Range-Doppler Maps (RDMs) from OFDM returns. Performance metrics of the JRC system are compared with standalone radar and standalone OFDM implementations, highlighting trade-offs in resolution, sidelobe suppression, and signal-to-noise ratio (SNR). While the 802.11p-based OFDM radar exhibited degraded resolution and PSLR due to limited bandwidth, a redesigned system using 320 MHz bandwidth at 77 GHz achieved significantly improved range resolution, making it viable for automotive applications. Despite slight compromises in radar resolution and communication capacity, the JRC system demonstrates its potential for multi-functional applications like autonomous vehicles, balancing performance with cost and spectral efficiency.
\end{abstract}

\section {Introduction}
     Vehicle-to-Vehicle (V2V) networks seek to enhance road safety and reduce traffic congestion by sharing road and traffic information between vehicles in real time. To avoid delays associated with third party networks, vehicle-to-vehicle communication systems need dedicated bandwidth \cite{8500189}. Instead of wasting additional spectrum resources, joint radar communication (JRC) systems attempt to integrate radar and communication systems \cite{8972666}. This results in reduced bandwidth usage and reduced power consumption while eliminating the need for additional hardware.
     
     In this paper, we design a JRC system, which uses OFDM for communication and zero-forcing to generate a range response from the OFDM returns. Our paper is organized as follows. First, we provide background and performance results for an radar automotive radar system. Next, we provide background and performance results for an OFDM 802.11p communication system. Finally, we create a JRC system based on the OFDM 802.11p standard and evaluate its performance against standalone radar and standalone OFDM implementations.
        
  \section {Radar}
   \subsection {Background}
   
In this section, we provide background on frequency-modulated continuous wave (FMCW) radar, which is used in many automotive applications. FMCW radar continuously transmits a signal whose frequency varies linearly over time (chirp). It measures the time delay between the transmitted and the reflected signals (echoes) to determine the distance of reflecting objects. It can also determine the relative velocity of these objects, by measuring frequency shifts caused by the doppler effect \cite{7455584}.

	\begin{figure}[H]
    		\centering
    		\fbox{\includegraphics[width=0.9\linewidth]{FMCW Blockdiagram}}
    		\caption{Typical FMCW Block Diagram \cite{9613183}}
    		\label{fig::fmcw_radar}
	\end{figure}
	
 The block diagram shown in Figure \ref{fig::fmcw_radar} depicts the signal processing chain of a typical FMCW radar system. The transmitter generates a frequency-modulated chirp signal of the following form:
 
 	\begin{equation}
 		x(t) = e^{j\pi{\beta}t^2/\tau}
 		\label{eq::chirp}
 	\end{equation}
 	
 	where $\beta$ is the chirp bandwidth and $\tau$ is the chirp duration.
 	
	It transmits this signal out of the antenna towards objects of interest. The receiver receives reflections from each of these objects, which are passed through a low noise amplifier. For each return, the received signal will be a chirp waveform with a frequency offset as illustrated below.
	
	\begin{equation}
		r(t) = e^{j\pi{\beta}(t - t_0)^2/\tau} = e^{j\pi{\beta}t^2/\tau}e^{-j2\pi({\beta}t_0/\tau)t}e^{j\pi{\beta}t_0^2/\tau}
		\label{eq::delayed_chirp}
	\end{equation}
	
	We can mix the received signal with the transmitted signal to create a tone for each return. The frequency of this tone will be proportional to its delay.
	
	\begin{equation}
		f = \frac{{\beta}t_0}{\tau} \Rightarrow t_0 = \frac{{\tau}f}{\beta}
	\end{equation}
	
	The delay of this return is then related to the range of the reflecting object as follows:
	
	\begin{equation}
		R = \frac{ct_0}{2}
	\end{equation}
	
	The relationship between the transmitted and received chirps signals is illustrated below:
	
	\begin{figure}[H]
    		\centering
    		\fbox{\includegraphics[width=0.8\linewidth]{FMCW Freq-Time graph}}
    		\caption{Plot of Transmit and Receive Signal Frequency vs Time.\cite{Long2019AssistingTV}}
    		\label{fig::fmcw_spectrogram}
	\end{figure}
	
	Note that the chirp bandwidth $\beta$ given in Equation (\ref{eq::chirp}) is inversely proportional to the range resolution ${\Delta}R$.
	
	\begin{equation}
		{\Delta}R = \frac{c}{2\beta}
		\label{eq::range_resolution}
	\end{equation}
	
	Because fine range resolution is desired, $\beta$ is typically very large (on the order of 150MHz-500MHz). This would typically require very high sample rates. However, one of the benefits of an FMCW radar is that the ADC samples tones instead of a full bandwidth signal. Thus, we only need to sample at a rate high enough to capture returns at the largest range of interest \cite{richards-2005}. For an automotive radar, this may only be on the order of 100-200 m, which can significantly reduce the sample rates and in turn reduce the radar's cost.
	
	After low-pass filtering and sampling the signal, Figure \ref{fig::fmcw_radar} illustrates the next step in the signal processing (a fast-time FFT). This FFT detects the frequency of each tone, which is proportional to the object's distance. Thus, the fast-time FFT allows the radar to detect the distance of reflecting objects.
	
	The object's relative velocity results in a doppler shift:
	
	\begin{equation}
		f_D = \frac{2v}{\lambda}
	\end{equation}
	
	Note that this doppler shift is twice what it is for a communication system, due to the reflection's two-way path. The phase change due to the doppler shift is small during a single chirp, so we instead measure it over multiple chirps (pulses). This is done by taking a slow-time FFT across pulses as shown in Figure \ref{fig::fmcw_radar}.
	
	The rate at which chirps (pulses) are repeated is denoted as the PRF. The PRF limits the radar's unambiguous range $R_{ua}$ and unambiguous velocity $v_{ua}$.
	
	\begin{equation}
		R_{ua} = \frac{c}{2\cdot\text{PRF}}
		\label{eq::unambig_range}
	\end{equation}
	
	\begin{equation}
		v_{ua} = \frac{\lambda\cdot\text{PRF}}{4}
		\label{eq::unambig_velocity}
	\end{equation}
	
	If a return exceeds these limits, it will alias and be detected at another range or velocity.
	
	\subsection {Simulation and Performance}

In this section, we evaluate the performance of an FMCW radar written in MATLAB and configured with the following set of parameters:

	\begin{center}
	\begin{tabular}{|c|c|}
		\hline
		Parameter & Value \\
		\hline
		Carrier Frequency & 77 GHz \\
		\hline
		Sample Rate & 300 MHz \\
		\hline
		PRF & 200 kHz \\
		\hline
		Number of Pulses & 256 \\
		\hline	
	\end{tabular}
	\end{center}
	
	Metrics of interest include the range resolution, peak sidelobe ratio (PSLR), doppler tolerance, unambiguous range, unambiguous velocity, and received SNR.

	The range resolution is the minimum range separation required for the radar to distinguish two objects. We can compute it according to Equation (\ref{eq::range_resolution}) as:
	
	\begin{equation}
		{\Delta}R = 0.5 m
	\end{equation}
	
	We can measure the range resolution by placing two objects very close together and observing the range response. In Figure \ref{fig::range_resolution}, we show the range response of two objects separated by 1 m. 
	 
	\begin{figure}[H]
	    	\centering
	    	\fbox{\includegraphics[width=0.8\linewidth]{range_resolution.jpg}}
	    	\caption{Range Response of Two Closely Placed Targets}
	    	\label{fig::range_resolution}
	\end{figure}
		
	Clearly, the minimum range separation at which the radar can distinguish two objects is when the objects are separated by a single range gate. This implies a range resolution of 0.5 m. This is consistent with the theoretical range resolution.
	
	The peak sidelobe ratio (PSLR) is the ratio of the range response peak to its highest sidelobe. The range response of an FMCW radar is the FFT of a tone (i.e. a sinc function). Therefore, the PSLR can vary drastically with the location of the FFT samples. To get a more consistent PSLR measurement, we zero-pad the fast-time FFT. Doing so, results in the following range response:
	
	\begin{figure}[H]
	    	\centering
	    	\fbox{\includegraphics[width=0.8\linewidth]{pslr_no_window_zpad.jpg}}
	    	\caption{PSLR Measurement with Zero Padding in Fast-Time FFT}
	    	\label{fig::pslr_zpad}
	\end{figure}
	
	Examining the figure, we measure a PSLR of 13.27 dB, which is the expected PSLR measurement for a sinc function.
	
	Note that we can improve the PSLR by applying a window. However, this will also degrade the range resolution by increasing the range response's mainlobe width. First, we consider a Chebyshev window with an 80dB peak to sidelobe ratio. The resulting range response is shown in Figure \ref{fig::plsr_window}.
	
	\begin{figure}[H]
	    	\centering
	    	\fbox{\includegraphics[width=0.8\linewidth]{pslr_with_window.jpg}}
	    	\caption{PSLR Measurement with Chebyshev Window}
	    	\label{fig::plsr_window} 
	\end{figure}
	
	Examining the range response, we achieve the expected PSLR of 80 dB. However, the mainlobe width is also increased by a factor of 3.2, which in term degrades the range resolution by a factor of 3.2. Other windows can be selected, which may off a better trade-off between range resolution and PSLR. Consider a taylor window, for example. It achieves a PSLR of 30.44 dB while only degrading the range resolution by a factor of 1.5.
	
	\begin{figure}[H]
	    	\centering
	    	\fbox{\includegraphics[width=0.8\linewidth]{pslr_with_taylor_window.jpg}}
	    	\caption{PSLR Measurement with Taylor Window}
	    	\label{fig::plsr_taylor_window} 
	\end{figure}
	
	We can measure the doppler tolerance of the system, by examining the PSLR in the presence of uncompensated doppler. In Figure \ref{fig::doppler_tolerance}, we show the PSLR when an 80 dB Chebyshev window is applied while sweeping the uncompensated doppler from $-v_{ua}$ to $v_{ua}$.
	
	\begin{figure}[H]
		\centering
	    	\fbox{\includegraphics[width=0.8\linewidth]{doppler_tolerance.jpg}}
	    	\caption{PSLR Measurement with Uncompensated Doppler}
	    	\label{fig::doppler_tolerance} 
	\end{figure}
	
	Note that the PSLR is independent of the uncompensated doppler. This is because the uncompensated doppler simply adds to the frequency of the input tone. This can introduce range error through a process known as range-doppler coupling. However, as long as the doppler shift is in the first ambiguity, we can compensate for any introduced range error.
	
	The unambiguous range and velocity of the system can be computed according to Equation (\ref{eq::unambig_range}) and Equation (\ref{eq::unambig_velocity}). This results in the following:
	
	\begin{equation}
		R_{ua} = 750\ \text{m}
	\end{equation}
	
	\begin{equation}
		v_{ua} \approx 194.8\ \text{m/s}
	\end{equation}
	
	We can measure the ambiguous range and velocity by placing a reflecting object at ranges and velocities that exceed these limits. For instance, to measure the unambiguous range, we place an object at a range of 1250 m and examine the resulting range doppler matrix (RDM).
	
	\begin{figure}[H]
	    	\centering
	    	\fbox{\includegraphics[width=0.8\linewidth]{ambig_range.jpg}}
	    	\caption{Range Doppler Matrix for Object at Ambiguous Range}
	    	\label{fig::ambig_range} 
	\end{figure}
	
	Figure \ref{fig::ambig_range} shows the RDM for this ambiguous range. The radar measures a range of 500.15 m. For an ideal system, the measured range is given as follows:
	
	\begin{equation}
		R_m = R\ \text{mod}\ R_{ua}
	\end{equation}
	
	Our measured range is consistent with this result thereby confirming the theoretical unambiguous range.
	
	We can confirm the unambiguous velocity in a similar manner. To do so, we measure the velocity of an object traveling at 250 m/s.
	
	\begin{figure}[H]
	    	\centering
	    	\fbox{\includegraphics[width=0.8\linewidth]{ambig_velocity.jpg}}
	    	\caption{Range Doppler Matrix for Object at Ambiguous Velocity}
	    	\label{fig::ambig_velocity} 
	\end{figure}
	
	Figure \ref{fig::ambig_velocity} shows the RDM for this ambiguous velocity. The radar measures a velocity of -139.92 m/s. For an ideal system, the measured velocity can be computed as follows:
	
	\begin{equation}
		v_p = v\ \text{mod}\ 2v_{ua}
	\end{equation}
	
	\begin{equation}
		v_m = \begin{cases}
			v_p & 0 \leq v_p < v_{ua} \\
			v_p - 2v_{ua} & v_{ua} \leq v_p < 2v_{ua}
		\end{cases}
	\end{equation}
	
	Our measured velocity is consistent with this result.
	
	According to \cite{richards-2005}, the SNR at the receiver input is given by the following equation:
	
	\begin{equation}
		\chi = \frac{P_tG^2\lambda^2\sigma}{(4\pi)^3R^4kT_0\beta_nF_nL_sL_{\alpha}(R)}
	\end{equation}
	
	Each of the variables in the above expression are defined as follows:
	
	\begin{center}
	\begin{tabular}{|c|c|}
		\hline
		Variable & Definition \\
		\hline
		$P_t$ & Transmit Power \\
		\hline
		$G$ & Antenna Gain \\
		\hline
		$\lambda$ & Wavelength \\
		\hline
		$\sigma$ & RCS \\
		\hline
		$R$ & Distance to Object \\
		\hline
		$k$ & Boltzmann's Constant \\
		\hline
		$T_0$ & Temperature \\
		\hline
		$\beta_n$ & Noise Equivalent Bandwidth \\
		\hline
		$F_n$ & Noise Figure \\
		\hline
		$L_s$ & System Loss \\
		\hline
		$L_{\alpha}(R)$ & Atmospheric Loss \\
		\hline
	\end{tabular}
	\end{center}	
	
	If we consider only the contributions of the transmit power and noise equivalent bandwidth, we can derive the relationship among the remaining variables:
	
	\begin{equation}
		\chi \propto \frac{P_t}{\beta_n}
		\label{eq::snr_propto}
	\end{equation}
	
	Note that the SNR in the RDM is greater than the input SNR by a factor of $G_{sp}$, where $G_{sp}$ is the SNR gain due to signal processing. The signal processing gain for the FMCW radar is given by:
	
	\begin{equation}
		G_{sp} = G_{fast}G_{slow}
	\end{equation}
	
	Here, $G_{fast}$ is the SNR gain due to fast-time processing, and $G_{slow}$ is the SNR gain due to slow-time processing. The fast-time SNR gain is given as follows:
	
	\begin{equation}
		 G_{fast} = \frac{M_{fast}}{L_{fast}} = \frac{f_s/PRF}{L_{fast}}
		 \label{eq::fmcw_fast_time_snr_gain}
	\end{equation}
		
	where $f_s$ is the sample rate and $L_{fast}$ is the SNR loss due to the fast-time FFT window. Using an 80 dB Chebyshev window, we get the following for $G_{fast}$:
	
	\begin{equation}
		G_{fast} \approx 860.8422
	\end{equation}
	
	Similarly, for the slow-time SNR gain, we have
	
	\begin{equation}
		G_{slow} = \frac{M_{slow}}{L_{slow}} = \frac{\#PRI}{L_{slow}}
	\end{equation}
		
	where $L_{slow}$ is the SNR loss due to the slow-time FFT window. Using an 80 dB Chebyshev window, we get the following solution for $G_{slow}$:
	
	\begin{equation}
		G_{slow} \approx 146.4771
	\end{equation}
	
	Therefore, the overall signal processing SNR gain is given by:
	
	\begin{equation}
		G_{sp} \approx 126094
	\end{equation}
	
	Or equivalently,
	
	\begin{equation}
		 {G_{sp}}_{(dB)} \approx 51.0069\ \text{dB}
	\end{equation}
	
	If we input a signal with an SNR of 20 dB and measure the SNR in the RDM, we obtain a value of 70.01 dB, which is consistent with the expected signal processing gain.
		
     \section {Communication (OFDM)}
     
	 \subsection {Background}
	 
	 	In this section, we provide background on OFDM (orthogonal frequency-division multiplexing). OFDM is a form of multicarrier modulation. In multicarrier modulation, the total channel bandwidth is divided into multiple subchannels, so that each subchannel experiences relatively flat fading \cite{djordjevic-2017}. The subchannel spacing of OFDM differs from other forms of multicarrier modulation. OFDM modulates symbols on orthogonal subcarriers of the following form:
	 	
	 	\begin{equation}
	 		s_k(t) = e^{j2{\pi}kt/T_s}\text{rect}(t/T_s)
	 	\end{equation}
	 	
	 	where $T_s$ is the symbol duration.
	 	
	 	The orthogonality of the subcarriers allows the bandwidth of neighboring subchannels to be partially overlapped, improving spectral efficiency \cite{djordjevic-2017}.
	 	
	 	A block diagram of an OFDM system is given in Figure \ref{fig::ofdm_block_diagram}.
		
	 	\begin{figure}[H]
	    		\centering
	    		\fbox{\includegraphics[width=0.9\linewidth]{OFDM Block diagram}}
	    		\caption{OFDM block diagram.\cite {10.1007/978-981-16-2406-3_61}}
	    		\label{fig::ofdm_block_diagram}
		\end{figure}
		
		When a bit stream is received, it is modulated using the desired 2-D modulation scheme (PSK, QAM, etc...). Then, it is placed into an FFT bin corresponding to a data subcarrier. Non-data carriers are reserved for pilot and null subcarriers as illustrated in Figure \ref{fig::ofdm_subcarriers}.
		
		\begin{figure}[H]
	    		\centering
	    		\fbox{\includegraphics[width=0.9\linewidth]{OFDM subcarriers}}
	    		\caption{OFDM subcarriers. \cite {inproceedings}}
	    		\label{fig::ofdm_subcarriers}
		\end{figure}
		
		Typically, null carriers are placed at the edges of the band to reduce interference between neighboring channels. A null carrier is also typically placed at DC to avoid problems with DC offsets in the D/A and A/D \cite{802_11a_standard}. An IFFT then modulates each of the symbols with its respective subcarrier.
		
		To deal with the effects of multi-path, a cyclic prefix is added to each of the OFDM symbols. The cyclic prefix appends the last L samples of an OFDM symbol to the beginning of the OFDM symbol as illustrated in Figure \ref{fig::cylic_prefix}.
		
		\begin{figure}[H]
	    		\centering
	    		\fbox{\includegraphics[width=0.9\linewidth]{cyclic_prefix.png}}
	    		\caption{OFDM Cyclic Prefix \cite{djordjevic-2017}}
	    		\label{fig::cylic_prefix}
		\end{figure}
		
		At this stage, we can also add a window to reduce the out-of band spectrum. In the case of windowing, the first \newline $L/2 + L_{win}$ samples are appended to the end of the symbol and the last $L/2 + L_{win}$ samples are appended to the beginning of the symbol. This is illustrated in Figure \ref{fig::ofdm_windowing}.
		
		\begin{figure}[H]
	    		\centering
	    		\fbox{\includegraphics[width=0.9\linewidth]{ofdm_windowing.png}}
	    		\caption{OFDM Symbol with Windowing \cite{djordjevic-2017}}
	    		\label{fig::ofdm_windowing}
		\end{figure}
		
		Then a window is multiplied element-wise with the symbol. A commonly applied window is provided in \cite{djordjevic-2017}, and is given by:
		
		\begin{equation}
			w_1(t) = \frac{1}{2}[1 - \cos\pi(t + T_{win} + T_G/2)/T_{win}]
		\end{equation}
		
		\begin{equation}
			w_2(t) = \frac{1}{2}[1 - \cos\pi(t - T_{FFT})/T_{win}]
		\end{equation}
		
		\begin{equation}
			w(t) = \begin{cases}
				w_1(t), & T_1 \leq t < T_2 \\
				1.0, & T_2 \leq t < T_3 \\
				w_2(t), & T_3 \leq t < T_4
			\end{cases}
		\end{equation}
		
		where $T_1$, $T_2$, $T_3$, and $T_4$ are defined as follows:
		
		\begin{equation}
			T_1 = -T_{win} - T_G/2
		\end{equation}
		
		\begin{equation}
			T_2 = -T_G/2
		\end{equation}
		
		\begin{equation}
			T_3 = T_{FFT} + T_G/2
		\end{equation}
		
		\begin{equation}
			T_4 = T_{FFT} + T_G/2 + T_{win}
		\end{equation}
		
		At the receiver, the window and cyclic prefix are removed before the signal is demodulated with an FFT. Because of the cyclic prefix, each of the received multi-path components are a circularly-shifted copies of the transmitted signal. After the FFT, the multi-path components result in a complex weight for each of the FFT bins. We can remove this complex weighting by equalizing the received symbol. To perform equalization, we can estimate the channel at each of the pilot carriers:
		
		\begin{equation}
			\hat{H}_{p_k} = \frac{Y_{p_k}}{X_{p_k}}
		\end{equation}
		
		We then can interpolate the channel estimate to get an estimate of the channel at each of the data carriers. Finally, we can compute equalization weights $A_k$ using either an zero-forcing or MMSE equalizer \cite{Vilar2015ImplementationOZ}.
		
		\begin{equation}
			A_{\text{ZF}_k} = \frac{1}{\hat{H}_k}
		\end{equation}
		
		\begin{equation}
			A_{\text{MMSE}_k} = \frac{\hat{H}_k^{*}}{|\hat{H}_k|^2 + 1/SNR}
		\end{equation}
		
		After equalization, we should be obtain to correctly reconstruct the symbols contained in the OFDM data carriers. Each of these symbols can then be demodulated to obtain the received bit stream.
		
\subsection {Simulation and Performance}
      
      We have developed comprehensive MATLAB code to evaluate the performance of an OFDM communication system. Metrics of interest include the peak to average power ratio (PAPR), the bit error rate (BER) in an AWGN channel, the BER in presence of Rayleigh fading, and the resulting capacity.
      
      For this analysis, we chose to evaluate an OFDM 802.11p system because the 802.11p standard is applicable to VNV communication. The 802.11p standard operates in the DRC Band (5.850 GHz to 5.925 GHz) and uses 64 subcarriers sampled at a rate of 10 MHz. A 16 sample cyclic prefix is added, increasing the symbol duration from 6.4 ${\mu}s$ to 8.0 ${\mu}s$. Of the 64 subcarriers, 48 are data carriers, and 4 are pilots. Pilots are specifically located at $\{-21, -7, 7, 21\}$. Data carriers cover all the remaining carriers in range $[-26, 26]$ except the DC subcarrier, which is a null carrier \cite{802_11p_abdelgader}.
      
    The power spectral density of the transmitted OFDM signal is shown in Figure \ref{fig::ofdm_psd}.
    
    \begin{figure}[H]
    		\centering
    		\fbox{\includegraphics[width=0.8\linewidth]{OFDM Power spectral density}}
    		\caption{OFDM Power Spectral Density}
    		\label{fig::ofdm_psd}
	\end{figure}
    
    The location of null carrier is also consistent with the 802.11p standard. We were able to perform power measurements on this signal to determine the PAPR. When the data carriers are modulated using QPSK modulation, we measure a PAPR of 11.8 dB. The measured PAPR is significant because it forces us to run the high power amplifier at reduced power levels, which can degrade the transmitted power and in turn the SNR of the system \cite{4289157}.
    
    To evaluate the BER in an AWGN channel, we add gaussian noise to the transmitted signal to achieve a given SNR. Then, we process the resulting signal with a model of the OFDM receiver. After FFT processing, we can generate a constellation diagram from the resulting data carriers. Figure \ref{fig::rx_constellation_diagram} shows the received constellation diagram for an OFDM signal using QPSK modulation at 20 dB SNR.
      
      \begin{figure}[H]
		\centering
    		\fbox{\includegraphics[width=0.8\linewidth]{OFDM AWGN Constellation}}
    		\caption{OFDM Received Constellation after AWGN Channel}
    		\label{fig::rx_constellation_diagram}
  	  \end{figure}
    
    		The expected bit error rate (BER) for a single OFDM subcarrier is identical to the BER of a single carrier system, when the SNR of the given subcarrier is considered. For QPSK, this is approximately given in \cite{goldsmith_2013_wireless} as
		
		\begin{equation}
			P_b(\rho_b) \approx Q(\sqrt{2\rho_b})
		\end{equation}
		
		where $\rho_b$ is the SNR per bit. If the modulation is the same for each OFDM subcarrier, the BER is the mean BER for all active data carriers. In Figure \ref{fig::ofdm_awgn_ber}, we examine the BER of an OFDM system vs SNR.
		
		\begin{figure}[H]
			\centering
    			\fbox{\includegraphics[width=0.8\linewidth]{OFDM AWGN BER vs SNR}}
    			\caption{OFDM BER vs. SNR in AWGN channel}
    			\label{fig::ofdm_awgn_ber}
		\end{figure}
		
		The SNR of each active subcarrier is 64/52 times (or 0.9018 dB) greater that of the received signal because all the signal energy is concentrated in the active subcarriers. After accounting for this SNR gain, the resulting BER is equivalent to the expected BER. 
        
        We also examine the BER of the OFDM receiver in the presence of Rayleigh fading. To do so, we apply Rayleigh fading to the signal amplitude prior to applying additive white gaussian noise. We can then generate a received constellation map for each of the symbols modulated on the data carriers. For 20 dB SNR, the received constellation map is shown in Figure \ref{fig::ofdm_rayleigh_fading}.
        
      \begin{figure}[H]
		\centering
    		\fbox{\includegraphics[width=0.8\linewidth]{OFDM Constellation Diagram}}
    		\caption{OFDM Received Constellation Diagram after Rayleigh Fading}
    		\label{fig::ofdm_rayleigh_fading}
  	  \end{figure}
    
    		We can compare the received constellation to the one shown in Figure \ref{fig::rx_constellation_diagram}. Doing so, we observe that the clusters are not easily separable like they were previously, indicating a much higher BER. The average BER in the presence of fading is given in \cite{djordjevic-2017} as
    		
    		\begin{equation}
    			\bar{P}_b = \int_{0}^{\infty}{P_b(\rho)f(\rho)d\rho}
    		\end{equation}
    		
    		$f(\rho)$ is the distribution of the SNR due to Rayleigh fading and is given by
    		
    		\begin{equation}
    			f(\rho) = \frac{1}{\bar{\rho}}e^{-\frac{\rho}{\bar{\rho}}}
    		\end{equation}
    			
    		where $\bar{\rho}$ is the average SNR. In Figure \ref{fig::ofdm_fading_ber}, the measured BER in the presence of Rayleigh fading is compared to the expected BER.
    		
    		\begin{figure}[H]
			\centering
    			\fbox{\includegraphics[width=0.8\linewidth]{OFDM BER vs SNR}}
    			\caption{OFDM BER vs. SNR in Rayleigh Fading}
    			\label{fig::ofdm_fading_ber}
  	  	\end{figure}
  	  
  	  	Referring to the figure, the measured BER rate is consistent with the expected BER.
    	    	  
    	  Finally, we can analyze the channel capacity of the system. The channel capacity of an individual OFDM subcarrier is given in \cite{djordjevic-2017} as
    	  
    	  \begin{equation}
			C_{k} = B_{k}\text{log}_2\left(1 +\frac{ \text P_{k}\cdot|H_{k}|^2}{N_{k}}\right)
	    		\label{eq:ofdm_subcarrier_capacity}
	    	\end{equation}
	    	
	    	where Bk is the bandwidth of the subcarrier, Pk is the transmit power of the subcarrier, Hk is frequency response of the subcarrier, and Nk is the noise power of the subcarrier. For an AWGN channel with uniform power allocation, the channel capacity is given by:
	    	
	    	\begin{equation}
	    		C = NB_0\text{log}_2\left(1 + \frac{P_0\cdot|H_0|^2}{N_0}\right)
	    	\end{equation}
	    	
	    	where $B_0$, $P_0$ $H_0$, and $N_0$ are defined with respect to the 0-th subcarrier. Note that we can redefine this equation using the bandwidth of the active carriers, which we denote as $B$ and the SNR of the active carriers.
	    	
	    	\begin{equation}
	    		C = B\text{log}_2\left(1 + \text{SNR}\right)
	    		\label{eq::chan_capacity_simplified}
	    	\end{equation} 
	    	
	    For an 802.11p OFDM system, the bandwidth of the active subcarriers is $52/64$ times the input bandwidth. As previously stated, the SNR of the active carriers is $64/52$ greater than the received SNR due to all the energy being concentrated in the active subcarriers. A plot of the resulting capacity after accounting account for these factors is shown below:
	    
	    \begin{figure}[H]
	    		\centering
    			\fbox{\includegraphics[width=0.8\linewidth]{awgn_capacity.jpg}}
    			\caption{OFDM 802.11p AWGN Capacity}
    			\label{fig::ofdm_awgn_capacity}
	    \end{figure}
	    
	   	In practice, the spectral efficiency of the system is lower than the capacity because of the bandwidth occupied by the pilots and the cyclic prefix, which reduce the effective data rates.
    
    \section {JRC}
		\subsection {Background}

		In this section, we design a JRC system using zero forcing to generate RDMs from OFDM returns. To understand our implementation, we first provide some more additional on radar processing. The ideal range response is given by:
		
		\begin{equation}
			x[n] = \delta[n]
		\end{equation}
		
		The frequency response of the ideal range response is flat:
		
		\begin{equation}
			X[k] = 1,\quad 0 \leq k < N 
		\end{equation}
		
		We can generate the ideal range response from any known sequence $z[n]$, which has a non-zero frequency response (i.e. $Z[k] \neq 0,\ 0 \leq k < N$). To generate the ideal range response we can multiply $Z[k]$ element-wise with $H[k]$, where $H[k]$ is defined as follows:
		
		\begin{equation}
			H[k] = \frac{1}{Z[k]},\quad 0 \leq k < N
		\end{equation}
		
		After element-wise multiplication, the result is $X[k]$. Finally, we can apply an IFFT to $X[k]$ to generate $x[n]$, the ideal range response.
		
		If we instead receive a circularly-shifted copy of $z[n]$ (i.e. $y[n] = z[[n-m]_N]$), the FFT of the resulting sequence is given as follows:
		
		\begin{equation}
			Y[k] = Z[k]W_N^{km}
		\end{equation}
		
		If we multiply $H[k]$ element-wise with $Y[k]$, we obtain $W_N^{km}$, which results in a range response of $\delta[n-m]$ after IFFT processing. This is the desired result.
		
		In an OFDM system, each of the reflections will be a circularly shifted copy of the transmitted waveform as long as their delay does not exceed the cyclic prefix. To physically implement this system, we need to consider the last $N$ samples of each transmitted OFDM symbol as $z[n]$.
		
		\begin{figure}[H]
			\centering
    			\fbox{\includegraphics[width=0.8\linewidth]{ofdm_radar_waveform.png}}
    			\caption{OFDM Sample Designation}
    			\label{fig::ofdm_radar_waveform}
  	  	\end{figure}
		
		After transmitting, we need to wait $L$ samples before receiving. If the signal has no delay, our range response will be a delta function. However, if the delay of the signal ($m$) is in range $[1, L]$, we will receive a circular-shifted copy of the $z[n]$, and the range response will be a delayed delta function of the form $\delta[n-m]$. 
		
		We do not have to take an FFT of $z[n]$ to obtain $Z[k]$. Instead, we can get $Z[k]$ from the OFDM subcarriers prior to IFFT processing. If the cyclic prefix is placed at the start of the symbol as shown in Figure \ref{fig::cylic_prefix}, $Z[k]$ is identical to the IFFT input. If the window is divided evenly between the start and the back of symbol as shown in Figure \ref{fig::ofdm_windowing}, $Z[k]$ is the IFFT input multiplied elementwise with $W_N^{kL/2}$.
		
		In the 802.11p standard, not all of the subcarriers are non-zero, which prevents us from fully realizing this system. However, we can instead invert all of the non-zero subcarriers, when determining the FFT of our matched filter, $H[k]$. The resulting frequency response is shown in Figure \ref{fig::ofdm_radar_freq_resp_no_window}.
		
		\begin{figure}[H]
			\centering
    			\fbox{\includegraphics[width=0.8\linewidth]{ideal_frequency_response_no_window.jpg}}
    			\caption{Ideal Frequency Response with Zero Subcarriers}
    			\label{fig::ofdm_radar_freq_resp_no_window}
  	  	\end{figure}
		
		We can take an IFFT to determine the resulting range response. However, because the frequency response is no longer flat, we will not be left with a delta function. Instead we expect a sinc function minus a DC offset due to the missing DC subcarrier. The corresponding range response is shown in Figure \ref{fig::ofdm_radar_range_resp_no_window}.
		
		\begin{figure}[H]
			\centering
    			\fbox{\includegraphics[width=0.8\linewidth]{ideal_range_response_no_window.jpg}}
    			\caption{Ideal Range Response with Zero Subcarriers}
    			\label{fig::ofdm_radar_range_resp_no_window}
  	  	\end{figure}
		
		The sinc-like appearance of the range response is not readily apparent. However, we can zero-pad the IFFT to increase the resolution of the range response. We must perform this zero padding in the middle of the input sequence to preserve the relative positions of the non-zero subcarriers.
		
		\begin{figure}[H]
			\centering
    			\fbox{\includegraphics[width=0.8\linewidth]{ideal_range_response_no_window_zpad.jpg}}
    			\caption{Interpolated Range Response with Zero Subcarriers}
    			\label{fig::ofdm_radar_range_resp_no_window_zpad}
  	  	\end{figure}
  	  	
  	  	As was the case with the FMCW radar, we can reduce the sidelobe level with windowing. For the JRC system, the windowing needs to be applied in the frequency domain prior to the IFFT. Figure \ref{fig::ofdm_radar_freq_resp_with_window} shows the frequency response after a 80 dB Chebyshev window is applied.
  	  	
  	  	\begin{figure}[H]
			\centering
    			\fbox{\includegraphics[width=0.8\linewidth]{ideal_freq_response_with_window.jpg}}
    			\caption{Ideal Frequency Response after Windowing}
    			\label{fig::ofdm_radar_freq_resp_with_window}
  	  	\end{figure}
  	  	 
  	  	The corresponding range response is shown in Figure \ref{fig::ofdm_radar_range_resp_with_window}.
  	  	
  	  	\begin{figure}[H]
  	  		\centering
    			\fbox{\includegraphics[width=0.8\linewidth]{ideal_range_response_with_window_zpad.jpg}}
    			\caption{Ideal Range Response after Windowing}
    			\label{fig::ofdm_radar_range_resp_with_window}
  	  	\end{figure}
  	  	
  	  	Note that the missing DC subcarrier severely limits the peak to sidelobe performance. As previously mentioned, the DC subcarrier has been nulled to avoid problems with DC offsets in the D/A and the A/D. It is possible to insert a non-zero DC subcarrier to get around this performance limitation. However, the OFDM receiver will need more complex IF sampling to take advantage it. As an alternative the transmitter can use the DC subcarrier and the OFDM receiver can ignore it. Note that this will degrade the capacity because there will be less power on the utilized subcarriers. Figure \ref{fig::ofdm_radar_range_resp_with_zero_subcarrier} shows the range response with an 80 dB Chebyshev window after adding a non-zero DC subcarrier.
  	  	
  	  	\begin{figure}[H]
  	  		\centering
    			\fbox{\includegraphics[width=0.8\linewidth]{ideal_range_response_with_zero_subcarrier.jpg}}
    			\caption{Ideal Range Response with Added DC Subcarrier}
    			\label{fig::ofdm_radar_range_resp_with_zero_subcarrier}
  	  	\end{figure}
  	  	
  	  	Note that the range response is now comparable to an FMCW radar. To determine the object's velocity, we can stack up many symbols and perform a Doppler FFT across them. This is the same process we used in the FMCW radar with symbols instead of chirps. A block diagram of the resulting receiver is shown in Figure \ref{fig::ofdm_radar_receiver}.
  	  	
  	  	\begin{figure}[H]
  	  		\centering
    			\fbox{\includegraphics[width=0.9\linewidth]{ofdm_radar_receiver.png}}
    			\caption{OFDM Radar Receiver}
    			\label{fig::ofdm_radar_receiver}
  	  	\end{figure}
  	  	
  	  	The unambiguous range and velocity of this system are given by Equation (\ref{eq::unambig_range}) and Equation (\ref{eq::unambig_velocity}), where the PRF is defined as follows:
  	  	
  	  	\begin{equation}
  	  		PRF = \frac{f_s}{N + L}
  	  	\end{equation}
  	  	
  	  	Here $f_s$ defines the sample rate, $N$ defines the number of subcarriers, and $L$ defines the cyclic prefix length.

\subsection {Simulation and Performance}
      
In this section, we analyze the radar and communication performance of the proposed JRC system. We start by investigating the radar performance of an unmodified OFDM 802.11p transmitter. For each shortcoming in performance, we modify the design to better meet the radar objectives. Then, we evaluate the change in OFDM performance.

We start this process by comparing the PSLR of the JRC system to the FMCW radar. The OFDM 802.11p transmitter includes a null subcarrier at DC which limits the PSLR as shown in Figure \ref{fig::pslr_802_11p_ofdm_radar}.

\begin{figure}[H]
\centering
\fbox{\includegraphics[width=0.8\linewidth]{802_11p_ofdm_range_response.jpg}}
\caption{PSLR Measurement for 802.11p OFDM Radar}
\label{fig::pslr_802_11p_ofdm_radar}
\end{figure}

There we measure a PSLR of 26.30 dB - even when using a 80 dB Chebyshev window. If we add a DC subcarrier to the OFDM transmission as previously suggested, we can achieve a much better PSLR as shown in Figure \ref{fig::pslr_802_11p_mod_ofdm_radar}.

\begin{figure}[H]
\centering
\fbox{\includegraphics[width=0.8\linewidth]{802_11p_ofdm_mod_range_response.jpg}}
\caption{PSLR Measurement for Modified 802.11p OFDM Radar}
\label{fig::pslr_802_11p_mod_ofdm_radar}
\end{figure}

The PSLR is now 79.92 dB, which is consistent with our 80 dB Chebyshev window. As mentioned above, the performance of the OFDM receiver will only degrade if it cannot take advantage of the added DC subcarrier. In this case, the capacity will degrade because less power will be allocated to the remaining subcarriers.

With the added DC subcarrier, we can now examine the doppler tolerance of the system. This is done by measuring the PLSR while varying the amount of uncompensated doppler. The doppler tolerance of the system is shown in Figure \ref{fig::ofdm_radar_doppler_tolerance}.

\begin{figure}[H]
\centering
\fbox{\includegraphics[width=0.8\linewidth]{OFDMRadar_PSLR vs Velocity}}
\caption{OFDM Radar Doppler Tolerance}
\label{fig::ofdm_radar_doppler_tolerance}
\end{figure}

Note unlike the FMCW radar, the PSLR varies with uncompensated doppler. With an 80 dB Chebyshev window applied, we see a PSLR degradation of about 25 dB for objects with a relative velocity of 100 m/s. The PSLR degrades with velocity, because uncompensated doppler creates a mismatch between the FFT of the matched filter, $H[k]$, and the FFT of the data, $X[k]$.

The JRC system is also only designed to detect returns with delays equal to or smaller than the length of a cyclic prefix. For the 802.11p standard, this range is defined as follows:

\begin{equation}
R_{max} = \frac{ct_0}{2} = \frac{c(L/f_s)}{2} = 240 m
\end{equation}

In Figure \ref{fig::ofdm_radar_pslr_vs_range}, we examine the PSLR against range.

\begin{figure}[H]
\centering
\fbox{\includegraphics[width=0.8\linewidth]{OFDMRadar_PSLR vs Range}}
\caption{OFDM Radar PSLR vs Range}
\label{fig::ofdm_radar_pslr_vs_range}
\end{figure}

As expected, we see that the PSLR steeply degrades when the range exceeds $R_{max}$. For comparison, we note that this range degradation does not occur in an FCMW radar.

For an OFDM radar, we have already stated the unambiguous range and velocity. However, due to the demonstrated performance degradations at large ranges and velocities, it is not meaningful to evaluate performance outside of the first range and velocity ambiguity. Even though it is limited to the first ambiguities, an automotive 802.11p radar will still be able to measure an acceptable range of ranges and velocities.

After adding a non-zero DC subcarrier, we can also examine the range resolution of the system. The range resolution is the minimum separation at which two objects are resolvable. For this experiment, we disable the range window to get the finest possible range resolution and zero-pad the IFFT to increase our visibility of the resulting range response. In Figure \ref{fig::ofdm_radar_range_resolution}, we show the range response of two objects separated by 30 m.

\begin{figure}[H]
\centering
\fbox{\includegraphics[width=0.8\linewidth]{802_11p_ofdm_range_resolution.jpg}}
\caption{802.11p OFDM Radar Range Resolution Measurement}
\label{fig::ofdm_radar_range_resolution}
\end{figure}

Examining the figure, we can infer that the object separation is approaching the radar's range resolution. For a sinc response, two objects are separated by the range resolution when one response reaches its peak, while the other experiences its first null. In other words the range resolution is half the range response's mainlobe width. In Figure \ref{fig::ofdm_radar_mainlobe_width}, we measure the mainlobe width of the range response without a window function.

\begin{figure}[H]
\centering
\fbox{\includegraphics[width=0.8\linewidth]{802_11p_ofdm_range_response_mainlobe_width.jpg}}
\caption{802.11p OFDM Radar Range Response Mainlobe Width}
\label{fig::ofdm_radar_mainlobe_width}
\end{figure}

Examining the above figure, we find that the mainlobe width is $138.01\ \text{m} - 101.824\ \text{m} = 36.186\ \text{m}$. Our measured range resolution is then $36.186\ \text{m} /2 = 18.093 \ \text{m}$. We can also compute the range resolution of the system using Equation (\ref{eq::range_resolution}) with a bandwidth $\beta = 53/64 \times 10\ \text{MHz} = 8.281\ \text{MHz}$. This results in a range resolution of $18.1\ \text{m}$, which is consistent with the measured data.

Note that this range resolution is insufficient for automotive radar applications. However, we can design an OFDM system with more bandwidth in the 77 GHz band that will achieve better range resolution. 

For this system, we increase the bandwidth from 10MHz to 320MHz and scale the OFDM parameters by 32. The updated parameters are included in the table below:

\begin{center}
\begin{tabular}{|c|c|}
\hline
\textbf {Parameter} & \textbf {Value }\\
\hline
Carrier Frequency & 77 GHz \\
\hline
Sample Rate & 320 MHz \\
\hline
Number of Subcarriers & 32x64\\
\hline
Number of Data Carriers & 32x48 + 1 \\
\hline 
Number of Pilot Carriers & 32x4 \\
\hline 
\end{tabular}
\end{center}

With the updated parameters, the active carriers will have a bandwidth of $320 \text{MHz} \times (52 \times 16 + 1)/(64 \times 16) \approx 260\ \text{MHz}$. This bandwidth should correspond to a range resolution of 0.5758 m. In Figure \ref{fig::802_11p_ofdm_improved_range_response_mainlobe_width}, we measure the mainlobe width of the range response to confirm the updated range resolution.


\begin{figure}[H]
\centering
\fbox{\includegraphics[width=0.8\linewidth]{802_11p_ofdm_improved_range_response_mainlobe_width.jpg}}
\caption{Updated OFDM Radar Range Response Mainlobe Width}
\label{fig::802_11p_ofdm_improved_range_response_mainlobe_width}
\end{figure}

Referring to the above figure, the mainlobe width is now 1.1527 m, which corresponds to a range resolution of 0.5763 m. It is important to note that this range resolution will further degrade after windowing is applied. However, for automotive applications, this is a much more acceptable range resolution than what was provided by the 802.11p radar.

We can also compare the SNR of this system to the SNR of the FMCW radar. As mentioned in Equation (\ref{eq::snr_propto}), the SNR is proportional to the transmit power. For an OFDM system, the transmit power is reduced by the PAPR to ensure the amplifier operates in its linear region. This will lead to an SNR degradation of 11.8 dB with respect to an FMCW radar.

The SNR in the RDM will also be scaled by the signal processing SNR gain ($G_{sp}$). Compared to the FMCW radar, only the fast time signal processing gain will be different. If the transmitted symbols are encoded with constant amplitude modulation, then the fast time signal processing SNR gain is given as follows:

\begin{equation*}
	G_{fast} = \frac{N}{L_{fast}}
\end{equation*}

where $N$ is the number of subcarriers and $L_{fast}$ is the SNR loss due to the fast-time window. For the given system, with an 80 dB Chebyshev window the fast time SNR gain will be

\begin{equation*}
	G_{fast} \approx 1175.54
\end{equation*}

Using Equation (\ref{eq::fmcw_fast_time_snr_gain}) with a PRF of $320\ \text{MHz}/(80 \times 16) = 125\ \text{kHz}$, we compute $1469.56$ as the equivalent FMCW SNR gain. Note that FMCW SNR gain is greater than the OFDM Radar SNR gain by approximately $(N + L)/N$. For the specified set of parameters, that works out to be roughly 0.97 dB. Therefore, we expect to the SNR in the RDM to be approximately 13 dB lower for the JRC system than it is for the FMCW radar.

\iffalse
The next plot displays the signal with the same parameter without the DC null. The pick received power is almost the same as the signal received with DC null. However, the sidelobes and noise floor shifted down significantly (average of 17 dB lower). Lower sidelobes and noise floor is crucial to some applications with high scattering noise. By including DC offset, the available bandwidth is fully utilized, leading to better spectral efficiency. Each subcarrier contributes to the total system data rate. Utilizing the DC subcarrier adds additional capacity, increasing the total throughput of the system.

\begin{figure}[H]
\centering
\fbox{\includegraphics[width=0.8\linewidth]{OFDMRadar_No DCNull_50m}}
\caption{OFDM/Radar without DC null range plot}
\end{figure}

The previous results were simulated for a target located at 50m. We have increased the target range to 500 and 900m to see the received data quality and degradation in a long target range. As the plots indicate, in the 500m range which considers a long range for V2V radar applications, we still have a distinguishable target peak (PSLR 23dB). At 900m range the dynamic range degrades significantly to around 9dB. \par
\textbf{ Note: } For an effective presentation, we only display the analysis for signals without the DC Nulls.
\begin{figure}[H]
\centering
\fbox{\includegraphics[width=1.0\linewidth]{Mix_500-900m}}
\caption{OFDM/Radar No DC null range plot 500m (left) and 900m (right)}
\end{figure} 
To ensure a fair comparison between OFDM and FMCW radar, the PSLR versus range graph is presented. Up to a range of 200 meters, the PSLR remains high at 80 dB, indicating excellent radar detection performance. Beyond 200 meters, the PSLR decreases significantly, and by approximately 1200 meters, the range becomes ambiguous.
\begin{figure}[H]
\centering
\fbox{\includegraphics[width=1.0\linewidth]{OFDMRadar_PSLR vs Range}}
\caption{OFDM/Radar PSLR Measurement vs Range}
\end{figure} 
The plot below represents a Range-Doppler Map (RDM), which is a visualization used in radar systems to show the relationship between detected targets' range and relative velocity. A strong return is observed around 100 meters range and close to 50 m/s velocity. This suggests a stationary or slow-moving target with strong radar reflectivity. Other ranges (e.g., 300m, 700m) show some faint signals, but these are significantly weaker compared to the strong reflection near 100m. The resolution along the velocity axis can be further refined to detect smaller velocity variations in moving targets which can addressed by optimizing the OFDM parameter in the next step. 
\begin{figure}[H]
\centering
\fbox{\includegraphics[width=0.8\linewidth]{OFDMRadar_RDM}}
\caption{OFDM/Radar RDM Plot}
\end{figure} 
The same PSLR vs velocity evaluation was performed for Doppler and is plotted below. It is important to note that, unlike standalone radar, the PSLR is influenced by uncompensated Doppler. The low PSLR values for Doppler make target velocity estimation challenging beyond 200 m/s.
\begin{figure}[H]
\centering
\fbox{\includegraphics[width=1.0\linewidth]{OFDMRadar_PSLR vs Velocity}}
\caption{OFDM/Radar PSLR Measurement vs Velocity}
\end{figure} 

A key advantage of the OFDM technique is its flexibility to adjust parameters, enabling optimal performance tailored to the application and channel characteristics in various environments. We have adjusted the OFDM parameters to optimize radar performance while preserving high communication efficiency and capacity.
OFDM radar written in MATLAB and configured with the following set of parameters: 

\begin{center}
\begin{tabular}{|c|c|}
\hline
\textbf {Parameter} & \textbf {Value }\\
\hline
Carrier Frequency & 77 GHz \\
\hline
Sample Rate & 320 MHz \\
\hline
Number of Subcarriers & 32x64\\
\hline
Number of Data Carriers & 32x48 + 1 \\
\hline 
Number of Pilot Carriers & 32x4 \\
\hline 
Number of Pulses & 128 \\
\hline 
Target Range & 50 m \\
\hline 
Target Velocity & 50  m/s\\
\hline 
\end{tabular}
\end{center}


The graph below presents the simulation results using the parameters described above. While the noise floor has risen to 45 dB, the PSLR remains around 38 dB, similar to the unimproved simulation at a 50m range. Another significant improvement is the enhanced resolution of the detected peak, leading to more accurate range measurements.

\begin{figure}[H]
\centering
\fbox{\includegraphics[width=0.8\linewidth]{OFDMRadar_No DCNull_T50_Improve}}
\caption{OFDM/Radar range plot for 50m target}
\end{figure} 
The RDM results for the given parameters, shown in the graph below, demonstrate enhanced resolution with the newly optimized OFDM parameters.
\begin{figure}[H]
\centering
\fbox{\includegraphics[width=0.8\linewidth]{OFDMRadar_RDM_Optimized}}
\caption{OFDM/Radar RDM Plot}
\end{figure} 
\fi

We can also analyze the communication OFDM performance after the system is tailored to JRC applications. The power spectral density of the transmitted waveform is included in Figure \ref{fig::updated_ofdm_psd}.

\begin{figure}[H]
\centering
\fbox{\includegraphics[width=0.8\linewidth]{Pwr Density_Optimized OFDMRadar}}
\caption{Updated OFDM Power Spectral Density}
\label{fig::updated_ofdm_psd}
\end{figure} 

As anticipated, the total power density decreases by 30 dB, a direct consequence of an increased number of data carriers, which have been increased by a factor of 32 compared to the 802.11p OFDM standard.

We have also generated an updated BER curve to illustrate the performance in an AWGN channel. The updated curve is included in Figure \ref{fig::updated_ofdm_ber}

\begin{figure}[H]
\centering
\fbox{\includegraphics[width=0.8\linewidth]{BER vs SNR_Optimized OFDMRadar_AWGN}}
\caption{Updated OFDM BER vs SNR}
\label{fig::updated_ofdm_ber}
\end{figure}

Note that the BER is equivalent to the BER curve shown in Figure \ref{fig::ofdm_awgn_ber}. This occurs because the ratio of active carriers to subcarriers remains the same (i.e. the signal energy has the same concentration in the active subcarriers).

In practice, for a fixed signal power, the SNR will be lower when the bandwidth is wider because we have a greater noise bandwidth and a fixed power spectral density. This can be visualized by expressing capacity in the following form:

\begin{equation}
	C = B\text{log}_2\left(1 + \frac{P}{{N_0}B} \right)
\end{equation}

We can now consider values of $\frac{P}{N_0}$ that achieve different SNRs with a 10 MHz bandwidth and examine the effects of increased bandwidth when the transmit power and noise power spectral density are held constant. Figure \ref{fig::capacity_vs_bandwidth} shows the results of this experiment along with the corresponding SNRs with a 10 MHz bandwidth.

\begin{figure}[H]
\centering
\fbox{\includegraphics[width=0.8\linewidth]{capacity_vs_bandwidth.jpg}}
\caption{Effect of Bandwidth on Capacity}
\label{fig::capacity_vs_bandwidth}
\end{figure}

Examining the above figure, we can conclude that the channel capacity of the modified OFDM system will not degrade as a result of the increased bandwidth.  

\section {Conclusion}

In this paper, we presented a JRC implementation, which used an OFDM 802.11p transmitter and zero forcing to generate RDMs from OFDM returns. We, then, compared the performance of this system to that of an FMCW radar.

We observed that performance of the radar degraded with range and uncompensated doppler shifts. Despite this degradation, we inferred that the OFDM radar would provide acceptable PSLR performance over ranges and velocities encountered in an automotive environment. We also observed a degradation in SNR due to the high PAPR and cyclic prefix of OFDM symbols. More significantly, we found that the range resolution of the 802.11p radar was significantly worse than the FMCW radar, due to its low bandwidth. This makes the 802.11p radar impractical for automotive radar use.

To achieve better range resolution, we also experimented with a 320 MHz bandwidth JRC system operating in the 77 GHz band. This provided a sufficient range resolution, without degrading the OFDM performance in an AWGN channel.

With a reduction in the OFDM PAPR and an increase in the channel bandwidth, we believe that the presented OFDM radar will be able to sufficiently meet both radar and communication objectives, while reducing total bandwidth usage, power consumption, and amount of required hardware.


	\bibliographystyle {IEEEtran}
	\bibliography {References}
	
	%\nocite{yang_subcarrier_multiplexing}
	%\bibliography{sources}{}
   % \bibliographystyle{ieeetr}
  
\end{document}

